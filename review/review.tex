\documentclass{article}
\usepackage[utf8]{inputenc}
\usepackage[english]{babel}
\usepackage{titling}
\usepackage[backend=biber, style=numeric]{biblatex}
\usepackage[margin=0.5in]{geometry}

\bibliography{review}

\title{A literature review of machine learning models for train delay prediction}
\author{D. White, N. Al-Moubayed}
\date{\today}

\begin{document}
\begin{titlingpage}

\maketitle
\begin{abstract}

\textit{Context}: train delay prediction (TDP) 

Train delays impose a huge cost on both train operators and passengers. A preliminary analysis of historical delay attribution data released by Network Rail (NR) shows that
35 million minutes of delay were experienced by passengers in the 2018 - 2019 financial year. The prediction of train delays allows the rescheduling or re-routing of crews and rolling-stock,
the reduction in the the amount of delay, and improvement of information available to passengers, and subsequent better decision-making. 

\textit{Objective}: the goal of this work is to synthesise available research results to inform evidence-based selection of machine learning (ML) models for TPD.

\textit{Method}: relevant studies about ML techniques were gathered via a systematic literature review. Supporting contextual studies were also gathered.

\textit{Results}: 19 studies were selected. 13 distinct models are used: Bayesian networks, support vector regression, random forests, neural networks, fuzzy Petri nets, extreme learning machines, various forms of regression () 
The data used, and results obtained, are compared. 

\textit{Conclusion}: to do.

\end{abstract}
\end{titlingpage}

\tableofcontents
 
\clearpage

\twocolumn


\section{Research questions}

We are interested in answering the following research questions:

\begin{itemize}
	\item RQ1: What ML models are commonly used for TDP?
	\item RQ2: What exogenous data is used to improve the performance of those models? / big data analysis
	\item RQ3: What are the research challenges in TDP?
	\item RQ4: No idea
\end{itemize}

\section{Introduction}

\subsection{Delays}

A \textit{delay} may be defined as "the difference between the minimum, or unopposed, travel time, and the actual travel time or [as] the difference between the scheduled running time and actual running time" \cite{dingler_koening_sam_christopher_2010}.

A \textit{delay} is a "positive deviation between the realized time and scheduled times of [an] activity" \cite{cerreto_nielsen_harrod_nielsen_2016}. 

They are reduced by the provision of  \textit{slack} in railway timetables \cite{cerreto_nielsen_harrod_nielsen_2016}. Slack is the amount of time a train can delayed without the delay propagating. 

Mention relationship between slack, capacity, and so on and so forth.

Although precise terminology differs, the literature agrees that there are two principal classes of delay \cite{olsson_haugland_2004}: primary and secondary.

A \textit{primary} or \textit{exogenous} delay is "caused by external stochastic disturbances" \cite{oneto_fumeo_clerico_canepa_papa_dambra_mazzino_anguita_2016}. The causes of primary delays are varied and numerous \cite{berger_et_al_2011}, \cite{milinkovic_markovic_veskovic_ivic_pavlovic_2013}, \cite{nr_delay_causes}, \cite{cerreto_nielsen_harrod_nielsen_2016}, and many different classifications exist. For the purposes of this dissertation, the following classification is proposed:

\begin{itemize}
	\item \textit{Weather}: severe heat, flooding, landslips, leaves, snow, and ice
	\item \textit{Passenger}: prolonged alighting and boarding times, large flows
	\item \textit{Maintenance}: construction work, repair work
	\item \textit{Other}: accidents, vandalism, trespassing, fatalities, strikes, holidays
\end{itemize}

A large component of this dissertation is the effect of the inclusion of exogenous data on the predictive capability of various machine learning techniques: this classification broadly follows the sets of data that will
be considered. 

The relative importance of these factors is poorly studied. A limited study by \cite{olsson_haugland_2004} found that the punctuality of trains is correlated with the number of passengers, occupancy ratio, departure punctuality, and operational priority rules.

The number of passengers affects the \textit{dwell time}, the time "devoted to the loading and unloading processes of the train" \cite{san_mohd_masirin_2016}. For this dissertation, this refers to the alighting
and boarding of passengers. Dwell time is a key parameter of system performance, service reliability and quality \cite{puong_2000}. Passenger volume is considered a key factor influencing both dwell time \cite{san_mohd_masirin_2016} and punctuality \cite{olsson_haugland_2004}.

A \textit{secondary} (\textit{knock-on}, \textit{consecutive}) delay is "generated by operations conflicts" \cite{cerreto_nielsen_harrod_nielsen_2016}, i.e. primary delays. Secondary delays often affect
both the route on which the primary delay occurred and any connecting routes; delays 'cascade' as "trains, drivers, and crews aren't in the right place at the right time to run other services" \cite{nr_knock_on_delays},
or o trains are held according to waiting policies between trains \cite{berger_et_al_2011}.

Although further classifications of secondary delay exist \cite{daamen_goverde_hansen_2008}, the current level of detail will suffice for this dissertation.

Secondary delays cannot be exactly forecast \cite{berger_et_al_2011}, \cite{milinkovic_markovic_veskovic_ivic_pavlovic_2013} because they are influenced by multiple interacting factors: the severity of the primary delay, the timetable of the train, the infrastructure, and even the behaviour of the driver, who may drive faster than planned, or reduce dwell time at stations, in order to make up time.

The difficulty of doing so cannot be overstated. Predicting the occurence of a primary delay is easy in comparison.

The goal of this work is to synthesise available research results to inform evidence-based selection of ML techniques.

Siding: a short section of double track on a single track that allowing trains to meet or pass each other \cite{barbour_et_al_2019}. A meet or a pass is referred to as a \textit{movement}; movements are directed by human dispatchers. 

\subsection{Timescales}

There are several different timescales at which delays can be predicted, such as short-term (predicted using real-time operating data) and long-time (3 days to a week in advance), as in \cite{wang_zhang_2019};
tactical (in which models are applied to both timetabling and resource planning), and operational (in which models are used for the real-time prediction of train delays), as in \cite{markovic_milinkovic_tikhonov_schonfeld_2015}. This dissertation is concerned with the short-term / operational level, henceforth referred to as the \textit{real-time} level. 

\cite{nair_et_al_2019} merge the real-time (operational) and short-term \ long-term (non-operational) trains in their ensemble model by utilising different underlying models.

Models for real-time traffic have so far focused on overcoming the combinatorial complexity of train rescheduling, rolling stock and crew scheduling, and delay management \cite{kecman_corman_meng_2015}.
Real-time train delay prediction (RTTDP) models are \textit{online}, i.e. updated as data on train movements becomes periodically available.  Many different models have been proposed; they will be discussed later.

For scheduled train services, a trade-off exists between efficiently utilising the capacity of a railway network and improving the reliability and punctuality of train operations. Used 3.25 years of data. 

Extensive consultations with SMEs to arrive a set of factors with explanatory power. Started on 2 performance metrics; ended up on 80. Roughly 350 features for operational trains, and 70 for non-operational trains. Five types: train specific (train properties and real-time train state. Information directly influences the delay of each train in the network.
Infrastructure: static and RT information regarding statiosn, platofmrs, and tracks. Conveys infrastructure properties such as how busy a station is and how frequently delay happens at a given track.
Network-related: a class of features related to delays across the network. Important subset is track and platform occupation conflicts. Feature generation routine aims to determine the likelihood of such downstream 
conflicts along tracks and platforms by considering arrival times within narrow time windows. 
Connection: delay may be caused by connecting trains. Determined from data using a rule-based approach.
Exteranl features: calendar features, weather information, trainworks, maintenance, holidays.

Trained two families of RF. First for operational trains: a $n$-stop model $n = 1, 2, ... 10$. Limited to 15 trees per model; models for departure and arrival time prediction are trained separately. Increasing had no impact of quality. For static / non-operational trains: one for departure, one for arrival. 22 RF for both. Can be intepretated. 

Kernel regression: store a reference catalog of movements for each train. Forecast is then generated by a weighted sum of the reference catalog. Weights are computed by measuring the similarity between the train of interest and the weighted set. Only used for operational trains. Tested for non-operational trains, but it performed poorly. 

Catalog for thousands of services. Caching service used (threshold-based heuristic). Discard forecasts generated by catalogs with fewer than 100 trajectories.

And a simulation model! Based on Szabo et al (2017). Run every minute, initialised by setting the train position to the most recent status message received in near real-time. Could not integrate data on delays caused by prolonged boarding and egress processes. Connections are accounted for by estimating a connection matrix of all possible pairs of trains at each station. If deemed significant, waiting time threshold is estimated. 

7400 MR models were needed daily. 

Edge case when trains with long delays did not report position, as stuck between control points. Results suggest that greatest potential for improvements is in shorter-term forecasts of operatiaonl trains using DDM. 

\subsection{Metrics}

\textit{Punctuality} is "a feature consisting in that a predefined vehicle arrives, departs, or passes at a predefined point at a predefined time" \cite{rudnicki_1997}. This definition has the interesting effect that trains 
that arrive \textit{early} cannot be considered punctual. However, the use of punctuality hides a lot of information \cite{skagestad_2004}; reliability and variability are better metrics \cite{olsson_haugland_2004}.

\textit{reliability} has several measures \cite{rietveld_bruinsma_van_vuuren_2001}:

\begin{itemize}
	\item The probability that a train arrives $x$ minutes late (punctuality)
	\item The probability of an early departure
	\item The mean difference between the expected arrival time and the scheduled arrival time
	\item The mean delay of an arrival given that one arrives late
	\item The mean delay of an arrival given that one arrives more than $x$ minutes late
	\item The standard deviation of arrival times
\end{itemize}

\textit{variability} is a "measurement of the uncertainty of trip journey times in transportation" \cite{olsson_haugland_2004}. It relates to the distribution of arrival times for a train \cite{noland_polak_2002}:
a train that arrives the same amount of minutes behind schedule every day has low variability, but not would be considered punctual.

The Office of Road and Rail (ORR), the UK's railway regulator, uses Public Performance Measure (PPM) to assess punctuality, and Cancellations and Significant Lateness (CaSL) to assess reliability.

A systematic literature review (SLR) is "a means of evaluating and interpreting all available research relevant to a particular research question or topic area or phenomenon of interest" \cite{williams_hollingsworth_2005}. The specific objectives of this SLR are:

\begin{itemize}
	\item To identify categories of ML techniques
	\item To summarise current research solutions for TDP
	\item To synthesis the current results from ML techniques for TDP
	\item To identify the research challenges and needs in the area of TDP
\end{itemize}

% LTIERATURE REVIEW METHOD
\clearpage
\section{Overview of literature review\\method}

A full systematic literature review is beyond the scope of this paper. 
Instead, a recent literature review focusing on big data analytics (BDA) in railway transportation systems (RTS) \cite{ghofrani_et_al_2018} was used as a basis for the authors' own.
The methodology of \cite{ghofrani_et_al_2018} is therefore thoroughly discussed. 
The studies identified in \cite{ghofrani_et_al_2018} are used to select key search terms for further database queries. Additionally, the cited references of those studies were used to find other relevant papers.
The studies found in \cite{ghofrani_et_al_2018}, by database search, and from cited references, are then selected for inclusion in this review by title, abstract, and finally by content.
Although the search was by no means exhaustive, the authors are satisfied that the papers gathered represent a comprehensive review of the application of ML to TDP.

\subsection{Recent applications of big data analytics in railway transportation systems: A survey \cite{ghofrani_et_al_2018}}

Readers are invited to read the paper themselves; only an overview of content relevant to this literature review is presented here.
The authors identified the following data-related keywords: "Data analytics, Big data, Data mining, Machine learning, Descriptive analytics, Predictive analytics" and the following RTS-related keywords: "Rail, Railway Engineering, Railway Systems, Railway Operations, Railway Safety, Railway Maintenance". Of particular significance is "Rail Operations", which covers the actual \textit{running} of trains on a RTS, and therefore delays.
The authors limited the scope of their search to papers in scientific journals, conferences, and dissertations in English from the last 15 years (i.e. 2003 - 2017). 
The authors specifically included only papers with quantitative results, disregarding those about qualitative challenges of BDA in RTS or purely mathematical modelling of RTS problems.
The authors searched ScienceDirect, Emeralds, Scopus, EBSCO, and IEEE Xplore, and also used cited references of studied papers as a source. 
115 papers were found and were classified by a four-layer structure:

\begin{enumerate}
	\item Area of RTS: Maintenance, Operations, Safety
	\item Analytic category: descriptive, predictive, prescriptive
	\item BDA model: clustering, numeric prediction, association, statistical analysis, image processing, simulation, classification, semantic analysis, text analysis, optimisation
	\item Implementation technique: Bayesian network, SVM, SVR, Decision Tree, ANN, Regression
\end{enumerate}

We are interested in Operations; papers in the other areas are disregarded. Within Operations, the authors discuss the applications of BDA to RTS, data collection and sources in RTS, and finally the studies themselves.
Only those that focus on TPD are considered. In total, 19 papers were selected by title for inclusion in this literature review; they provided the foundation for a subsequent database search.

\subsection{Database search}

The papers selected from \cite{ghofrani_et_al_2018} were used to identify the following key search terms: "train", "delay" and "prediction". Alas, "train" is a common word in scientific literature, and this confounded initial results somewhat. Appropriate synonyms were identified for each term. Where databases allowed Boolean operators, the following formulations were used:
\begin{itemize}
	\item (((trains OR train) NOT training) OR rail OR railway OR railways OR railroad OR railroads)
	\item (delay OR delays OR "event times")
	\item (prediction OR predicting OR analysis OR analyzing OR estimation OR estimating)
\end{itemize}

The following databases, based on those used by \cite{heckman_williams_2011}, were searched:

\begin{itemize}
	\item ACM Digital Library
	\item IEEE Xplore
	\item ScienceDirect
	\item SpringerLink
\end{itemize}

Where possible, the discipline was restricted to Computer Science. Initially, approximately 3000 studies were identified. Some post-processing was necessary to reduce the number of studies retrieved from IEEE Xplore, in particular, resulting in 69 studies.

%ScienceDirect yielded 73 articles, of which 44 were selected by title. 
%SpringerLink yielded 119 articles, of which 11 were selected by title.
%ACM yielded 9 articles, of which 5 were selected by title.
%IEEE Explore yield 22 articles, of which 9 were selected by title.

\subsection{Study selection}

Study selection was a three-stage process:

\begin{enumerate}
	\item Initial selection by title
	\item Selection by abstract
	\item Exclusion by content
\end{enumerate}

Of the 88 total studies found, 5 were duplicates. Studies were further selected by abstract. If the authors were content from the abstract a paper was relevant, it was selected; otherwise, the paper was read to determine suitability. In total, 19 studies were identified for inclusion in this literature review. Of those, 4 were discovered through other means - either from a prior, less structured, search, or from cited references.

% STUDIES
\clearpage
\section{Overview of studies}

We identified 19 studies in the literature that focus on the application of ML to TPD. An preliminary analysis shows that all work occurred in the past decade (i.e. during, or after, 2010), with most studies (47\%) published in 2017 and 2019.

\begin{table}[h]
\centering
\begin{tabular}{ll}
\noalign{\smallskip}\hline \noalign{\smallskip}
Year  & \# \\	\noalign{\smallskip}\hline \noalign{\smallskip}
2010  & 1  \\
2011  & 1  \\
2012  & 1  \\
2014 & 1	\\
2015  & 2  \\
2016  & 2  \\
2017  & 4  \\
2018  & 3  \\
2019 & 5  \\ 	\noalign{\smallskip}
Total & 19 \\  \noalign{\smallskip}\hline
\end{tabular}
\end{table}

13 distinct ML techniques were identified. 29 were applied in total, as several papers compared and contrasted different techniques, or variations of the same technique (as in \cite{lessan_fu_wen_2019}\cite{oneto_fumeo_clerico_canepa_papa_dambra_mazzino_anguita_2016}\cite{milinkovic_markovic_veskovic_ivic_pavlovic_2013}\cite{markovic_milinkovic_tikhonov_schonfeld_2015}), or even combined multiple models (e.g. the random forests regression of \cite{wen_et_al_2017}, or emsemble model of \cite{nair_et_al_2019}; in these cases, each distinct 'usage' is counted separately. 

The most popular techniques are random forests (20\%) and extreme learning machines (17\%)., although these figures are skewed by the inclusion of four papers CITE HERE which are closely related.

Some studies use techniques that may be more accurately classified as 'statistics' rather than ML (i.e. simple variants of regression, as in \cite{pongnumkul_pechprasarn_kunaseth_chaipah_2014}\cite{wang_work_2015}) or which are closer to \textit{algorithmic} models than ML (\cite{hansen_goverde_van_der_meer_2010}); however, as these lines is blurred, and for completeness' sake, all were selected for inclusion in this review.

Many defy easy classification. RFR should be just RF. Dynamic interpretable should be under Ensemble.

\begin{table}[h]
\begin{tabular}{lll}
\noalign{\smallskip}\hline \noalign{\smallskip}
ML model & Acronym & \# \\ 	\noalign{\smallskip}\hline \noalign{\smallskip}
Bayesian network & BN & 4 \\
Kernel method & KN & 1 \\
Extreme learning machine & ELM & 5 \\
Random forest & RF & 6 \\
Fuzzy Petri net & FPN & 1 \\
Adaptive neural fuzzy inference system & ANFIS & 1 \\
Gradient-boosted regression trees & GBRT & 1 \\
$k$-nearest neighbour & $k$-NN & 1 \\
Artificial neural network & ANN & 2 \\
Support vector regression & SVR & 2 \\
Kernel regression & KR & 2 \\
Markov &  & 2 \\
Decision tree & DT & 1 \\ 	\noalign{\smallskip}
Total                                    &         & 29	\\ \noalign{\smallskip} \hline
\end{tabular}
\end{table}

\section{ML models}

In this section, each of the distinct ML models identified previously is discussed. It is worth taking studies which compare models with a hint of skepticism. Authors tend to invest considerably more time in the primary model explored; it is unsurprising, therefore, they perform better.

% Millie uses:
% objectives
% method limitations
% evaluation methodology
% evaluation subjects (dataset)
% evaluation results

% BAYESIAN NETWORKS
\subsection{Bayesian networks}

We shouldn't have specific sections for each. We're trying to group by model, remember. 
A Bayesian network (BN) is a "probabilistic graphical model that uses Bayesian inference for probability computations" \cite{towards_data_science_BN_intro}. Each directed edge models a conditional independence, allowing "the incorporation of massive historical data" \cite{lessan_fu_wen_2019}.

The study compared heuristic hill-climbing, primitive linear, and hybrid structure BNs.

There is no common dataset for TDP, unlike other ML areas such as computer vision. That said, data trends can be observed in the papers gathered. Several use the TNV-Extract tool developed by Goverde ? and thus use data from the Netherlands. Several use HSR data from China. Four - those use Italian rail data. The fields of each dataset are explored later on. 

\subsubsection{A hybrid Bayesian network model for predicting delays in train operations \cite{lessan_fu_wen_2019}}

Objectives

 introduced the first hybrid BN-based to the area of TDP. 

The hybrid heuristic BN, built using naive and heuristic structures and refined by domain knowledge and expert judgements.
Achieved 80\% accuracy over a 60-minute prediction horizon. 

Advantages: it is simple, and so computational efficient.
The authors note that results could be improved by including the 'spatiotemporal' properties of each section, which we have taken to mean the speed at which trains can run. 

The MAE prediction error was around 30s; the RMSE for both predicted arrival and departure delays was less than 2 minutes

However. Predictions from the hybrid model matched observations only 56\% of the time. The authors attribute this to primarily to the discrete prediction space, and so employed discretisation to convert continuous variables into bins.  

explore three different Bayesian network schemes: heuristic hill-climbing, primitive linear, and hybrid. Hybrid, incorporating domain knowledge and judgements of local experts, was found to outperform other models, with an accuracy of over 80\% in predictions within a 60-minute horizon. The authors define a railway system as several interconnected subsystems: infrastructure, rolling stock, control and communication, and various operational rules and policies. 
It was found that arrival and departure delays follow the same distribution, with a linear relationship (chain) with a high correlation between arrival and departure delays at the same station (at least 94%). 

\subsubsection{Stochastic prediction of train delays in real-time using Bayesian networks \cite{corman_kecman_2018}}

present a stochastic model for predicting the propagation of train delays based on Bayesian networks (BNs). BNs allow the updating of probability distributions and reduce the uncertainty of future train delays in real-time as more data continuously comes available from the monitoring system. This authors extend this approach by modelling the interdependence between trains that share the same infrastructure or have a scheduled passenger train. The model is tested on historical train realisation data from a bus corridor in Sweden

That is interesting. \cite{nair_et_al_2019} trained models on three (apparently) identical measures of delay: travel time, delay, and additional delay (delay accrued since the previous stop). They found that kernels based on the former two substantially outperformed those based on the latter. For computing additional delay, the reference set itself needed to be recomputed each time there was an observation. This proved computationally prohibitive, and so the final model implemented was on delays.

Data-driven methods may not extrapolate well to rare and extreme events such as heavy network disruptions, especially wen few or no examples exist in the training data \cite{barbour_et_al_2019}.
Trains are heterogenous in respect in tonnage, power, length, and priority. 

% NEURAL NETWORKS

Although initial TDP models used neural networks, their popularity has declined, and they are now used primarily as a benchmark against which to compare other, more sophisticated, ML models.


% SUPPORT VECTOR REGRESSION
\subsection{Support vector regression}

Support vector regression (SVR) uses support vector machines (SVMs) as

A SVM maximises the margin between two or more classes to find the optimal \textit{hyperplane}, the seperator between two classes. In SVR, the hyperlaine is used to 

SVR is a popular ML algorithm grounded in statistical learning theory and for which training is efficient due to the convexity of the training problem \cite{barbour_et_al_2019}.
Construct a vertex-edge graph from track infrastructure data. 

A kernel 

SVRs can be used for continuous values; SVMs are for classification.

\cite{nair_et_al_2019} evaluated SVRs as a potential candidate for inclusion in their EM. However, they were found not to provide the best accuracy, quadratic in training data volume, and very sensitive to hyper-parameters.

\cite{barbour_et_al_2018} use SVR to estimate freight delays. They use origin-destination specific features, train priority, and train counts as influencing factors. They do not report on prediction error, but show relative improvement over a baselines of historical mean forecasts". Uses real-time data. Uses track geometry: grade and curvature information, single and multi-track territory, length of sidings), historical runtime profiles of trains, properties of trains (length and tonnage) and crew records. Note the difficulty of using features that change: the amount of traffic on the line of the road, the number of available sidings (as trains enter and leave).

Construct distincit regression model for each origin-destination pair for which predictions are required. All of the same form; differ only in feature weights and hyper-parameters. \cite{oneto_et_al_2016} do the same; they note that approximately 600,000 models would need to be retrained each day.

In a railway network with $k$ nodes, at minimum $k^2$ models are required; possibly more if multiple paths exist between each origin-destination pair. In practice, the system is rather more complex. A railway network can be considered a collection of sub-networks, each a route or line (likely operated by a separate entity, as in the UK), and with little engagement between different lines.

The paper needed to train only 140 models, instead of 1225. The authors estimate a total of 10,000 models are necessary for all ETA predicitons. 

They predict arrival delays, and note an average improvement of 14\% across the study area. 

Collection of datasets: freight train movement, train car operations, crew, and locomotive data. Network data is extracted from dispatching, operations, and signalling data. 
The authors note the importance of crew data. Most countries have a maximum on-duty duration (12 hours in the US) after which they must legally go off duty, despite the large expense of stopping a train and transporting a replacement crew. 

However, freight trains are very different from passenger trains: fewer stops, longer journeys, and different times (usually overnight). Final features: train length, train tonnage, train horsepower per ton, train priority, crew time remaining, on duty time to departure, full traffic count, directional traffic count, available sidings. 

The authors test four different SVR-based algorithms: three linear, with varying input features, and one RBF (radial basis function) kernel SVR. RBF kernel offers no improvement over fully-featured linear kernel. 

Large gains from including track segment occupancy features. Results compare closer to a deep NN trained on the same dataset (average improvement of 16\%, max of 25\%)
RBF kernel SVR offers a mean 14.3\% improvement, a max 21.8\%.

\cite{markovic_et_al_2015} found that their SVR performed better than the ANN. First application of an SVR to TDP. 

The paper considers only seven influencing variables:
Passenger train category (suburban, regional, long-distance)
Scheduled time of arrival at station (continuous)
Infrastructure influence defined by expert opinions (3 – 9)
Percent of journey completed distance-wise (continuous)
Distance travelled (continuous)
Time travelled (continuous)
Headway (continuous)

Found that scheduled time of arrival and headway are not strongly correlated with any other covariates. Compared an ANN and CVR. Categorical variables were converted to binary variables, Trained using Levenberg-Marquardt backprop. 100 independent ANNs were trained and the outputs averaged. 

727 passenger trains (99 long-distance, 321 regional, 307 suburban, northbound towards Belgrade. Delays recorded on a minute scale. 

Authors state an interesting extension would be the capture of infrastructure influence varibales thourgh input variables without quantification from SMEs. Or further stratification of data: seperate models for short and long delays, as they are caused by different things. 

Capturing the state of the network at a given point in time is the key thing here. Fascinating. Absolutely fascinating. 

% RANDOM FORESTS AND DECISION TREES
\subsection{Decision trees and random forests}

First introduced in \cite{ho_1995} (Random Decision Forests)

A random forest (RF) is a collection of individual decision trees (DT). Simply put, each individual tree predicts the class of an input and the class with the most votes is the output of the model. "A large number of relatively uncorrelated models (trees) operating as a committee will outperform any of the individual constituent models" (https://towardsdatascience.com/understanding-random-forest-58381e0602d2). This is the same that underpins ensemble methods; a RF could, in fact, be considered an ensemble method. 

Decision trees in which the target variable can take a discrete set of values are called classification trees; those that take continuous variables, regression trees. Leaves represent class labels and branches conjunctions of feature that lead to that class. 

A random forest is a meta-estimator. Fits several decisions trees on various subsamples of the training data and uses voting or averaging to improve accuracy and prevent overfitting. 

\cite{nabian_et_al_2019} uses a novel bi-level RF for TPD. The primary level predicts whether a delay will increase, decrease, or remain unchanged in a specified time frame; the second then estimates the actual delay (in minutes), given the predicted delay category at the primary level. Constructing the model is computationally cheap. 

Found that the proposed model provides the best prediction accuracy. Unusual structure is to meet recommendations of the two main railway operators in the Netherlands (ProRail and NS). Hence the coupled classification-regression task. 

Compared to linear regression, multi-nomial logistic regression, decision trees, $k$-NN, and SVM / SVRs. 

Slightly different focus: not the prediction of the occurrence of delays, but the prediction of change of \textit{severity} of a delay, given that it has already occurred (i.e. the train is already behind schedule). Includes the planned timetable, actual historical train performance, crew schedules, rolling stock circulation, (limited) infrastructure data: the distance between consecutive stations.
10 million data points over a 13 week period. Excluded Wednesdays (which are apparently particularly busy), delays longer than 15 minutes.

Extent to which drivers can compensate for the delay depends on the distance and planned time difference between stations. 
$l$ is the number of class labels in the classification task. Consists of a RF classifier and $l + 1$ RF regression models.

Assumed that all trains rode on a single track.

\cite{oneto_et_al_2018} (dynamic, interpretable) Focuses on the running time, dwell time, train delay, penalty cost (an interesting idea). Perhaps only applicable with different classes of train, right? and train overtaking between two trains in the wrong position relative on the railway network. Two approaches; one is based on the knowledge of the network and the experience of the operators. The other is based on the anaylsis of the historical data of the network with advanced data analytics methods. Former are interpretable and robust, but not very accurate; latter are the opposite. 

Former is not computationally demanding, hard to modify to incorporate complex phenomena, and not dynamic. 

Propose a hybrid model (HM). This perhaps should be under ensemble methods. Handles infrequent events (e.g. passage of freight trains), fast changes of train movements, easily extensible. First proposal of such a HM. Delayed if 30s late! 
Predict running time for all subsequent railway sections that it traverses. Dwell time for all the subsequent checkpoints at which it will stop, updating each time it reaches the next checkpoint. Allows TOCs to undestand how much time a train needs to complete the itinerary. 

Penalty cost. Fairly novel. NR has too has a ex post facto delay attribution process. Based on train category, operational category, type of railway section, amount of delay, and percentage of responsibility. 

A DDM similar to that described in \cite{oneto_et_al_2016} is currently in use at RFI, the Italian infrastructure manager (as in \cite{oneto_et_al_2018} (dynamic). The RFI DDM consists of many DDMS. For each train and each checkpoint, a DDM is constructed.  Can be built using many different learning algos. RF lead to better results.
If the timetable changes, it takes at least a month of data before achieving a reasonable accuracy. C

Idea of the EBM is to analyse time each train needs to traverse each seciton, based on speed limits, state of the network, and type of train. A coefficient (gaining time) is computed which represents the time that can be regained in case of delay. This is static. 

HM predict running time, dwell time, train delay, penalty cost, and trian overtaking. Encapsulates the experience of operators into a decision tree, but the leaves are constructed following ideas of the DDM. Does not implement one model per train. Groups trains on a series of similarity variables. Grouping increases the number of historical data to exploit during leaf creation, fixes new timetable issue. Leave is a DDM that learns from historical data of all trains that fall into that particular leaf (similar characteristics, iterinary). Each leaf is a RF. So really it's a fucking DT of RF!
The whole HM is constructed and updated incrementally as soon as nwe train movements are recorded in an online fashion. 

In the top-level decision tree, a new leaf is added each time a new train movement that belongs to a previously unexplored branch of the decision tree. The RF regressor in the leaf is trained based on all the past train movements that fall in that leaf. Forget movements older than 3 months. Based on experience of operators and different window size. Prediction is just consulting the appropriate leaf. 

Clearly outperforms the EBM and the DDM. Most evident for freight and regional trains. Constantly better across the whole year. Reaches optimal accuracy after 10 days of results.

Don't predict delays directly: instead use running time and dwell time predictors as building blocks. 

HM provides the best trade-off between accuracy and computational requirements. 

Lacking the co-operation of the NR. 

\cite{nair_et_al_2019} selected RFs for the accuracy of forecasts and the possibility of incremental (updating model parameters as fresh data becomes available) and parallel training. 

\cite{wen_et_al_2017} DD delay-recovery for HSR. Identifies main variables that contribute to delay: total dwell (TD), running buffer (RB) time, magnitude of primary delay (PD), and individual sections' recovery.

Evaluates multiple linear regression and random forest regression. 

Delay recovery: how fast a train service can recover from its delay in the subsequent stations. An important performance measure that shows the reliability and robustness of the service being provided. 

roughly 30,000 records. 14 stations. roughly 1100km of track. 

PDs can be reduced by adjusted running and dwell times. Buffer time = differenced between the scheduled running time (or dwell time) and the minimum required running time (or dwell time) is included in the schedule.

% REGRESSION
\subsection{Regression}

Regression is used to construct a relationship between two or more explanatory variables (independent) and a response (dependent) variable by fitting a linear eeuqation to the data. 
Found that around 200 trees was suitable. Ended up with 231 trees. Each tree hs two variables for each node.  For categories of 3 minutes, RFR was 90.9\% accurate; for MLR it was 84.4\%. 

Absolute mystery, this. 

% ENSEMBLE METHODS

\subsection{Ensemble methods}

Ensemble methods use multiple methods to generate forecast. The rationale behind such a framework is simple: gathering forecasts from a diverse set of models reduces bias and error rates. \cite{nair_et_al_2019} uses a purely data-driven ensemble method; \cite{oneto_et_al_2018} combine a RF and a experience-based model.

\cite{nair_et_al_2019} used 3 models for operational trains (those currently running): a RF ($n$-stop ahead), mesoscopic simulation and kernel regression.
And static RF for non-operational trains, and mesoscopic simulation.

\clearpage
\section{Analytical models}

A brief overview of analytical models is provided for context, with emphasis on the shortcomings of such models, and how data-driven approaches can ameliorate these shortcomings. In short: analytical models
perform worse, but are easy to explain, understand, and interpret. Analytical models cannot capture the complexity of such models. 

An \textit{analytical model} is "primarily quantitative or computational in nature and represents the system in terms of a set of mathematical equations that specific parametric relationships and their associated
parameter values as a function of time, space, and/or other system parameters" \cite{friedenthal_moore_steiner_2012}. Current state-of-the-art TDPS use analytical models \cite{oneto_fumeo_clerico_canepa_papa_dambra_mazzino_anguita_2016}.

Simplistic early models, such as \cite{frank_1966} made overly restrictive assumptions about railway operations by, for example, forbidding overtakes, assuming that departure times are uniformly distributed,
and that the speed of each train is unique and constant. 

Subsequent work in this area has largely relaxed these assumptions, by including factors such as overtakes, different speeds, priority systems, and uncertainties associated with train departure time \cite{petersen_1974}, \cite{chen_harker_1990}. More complex models have also emerged, incorporating stochastic approximation \cite{carey_kwiecinski_1994} and the impact of dispatching strategies on train delays
and passenger waiting time \cite{ozekici_sengor_1994}. Although the state-of-the-art advances constantly, a good example of an recent \textit{in-use} system is \cite{berger_et_al_2011}, which is currently used in the German rail network.

 % Datasets
\section{Datasets}

There is no common dataset for TDP, unlike other ML areas such as computer vision. That said, data trends can be observed in the papers gathered. Several use the TNV-Extract tool developed by Goverde ? and thus use data from the Netherlands. Several use HSR data from China. Four - those use Italian rail data. The fields of each dataset are explored later on. 

\section{Fields}

\section{Exogenous data}

It is widely accepted amongst ML practitioners that the greater the quantity of information available for the creation of a model, the greater the performance of that model will be. Features can either be \textit{engineered} or exogenous data can be incorporated. This is the realm of 'big data', which involves "multiple datasets and a complicated structure" \cite{Ghofrani_et_al_2018}. This section is broken down by the classification defined in the introduction.

Data is exogenous if it is independent of other input data but the output data depends on it. The scope for inclusion is essentially limitless: any source of data which may affect railway operations is a viable candidate. 
In the studies selected for this review, there are two main sources: infrastructure \cite{markovic_et_al_2015} \cite{milinkovic_et_al_2013} via \textit{expert opinion}, and weather \cite{oneto_et_al_2017} \cite{oneto_et_al_2018}, \cite{oneto_et_al_2019}. 

Special mention must go to \cite{nair_et_al_2019}, which used network traffic states, such as likely stretch conflicts and current headways, weather, event information, work zone information, inferred occupation conflicts, train connections, and rolling stock rotations.

% WEATHER
\subsection{Weather}

Weather is a common cause of primary delay. The classification defined earlier included severe heat, flooding, landslips, leaves, snow, and ice. It is expected that weather-induced delays are seasonal. Severe heat is likely to cause delays in summer; leaves in autumn; and snow and ice in winter. \cite{brazil_2017} found that most weather-caused delays occurred in the last third of the year, with a peak in November. Weather is also a popular inclusion for TDP. 

Weather was first included in a TDP model, to the best of the authors' knowledge, in \cite{oneto_et_al_2016}.

Subsequent studies have established the impact of severe weather on train delays "BRAZIL201769, title = "Weather and rail delays: Analysis of metropolitan rail in Dublin".
Papers largely agree that, dependent on climate, weather delays most trains during the last third of the year, with November a particular culprit, likely due to the sustained impact of leaf-fall.

\cite{wang_et_al_2019} observed that in locations less prepared for specific severe weather - such as snowy weather in southern cities - delays were greater. They found that in severe weather trains delays are determined by mainly the type of bad weather, but in ordinary weather they are determined mainly by historical delay time and the delay frequency of trains. 

Fields tend to be largely consistent: there are only so many weather variables of note. \cite{wang_et_al_2019} used lowest temperature, highest temperature, weather category (e.g. "overcast", "light rain"), Beaufort scale (wind speed) and air quality index (seemingly unique to China). Data was collected from 344 cities along the route in question. However, the timeframe used was between 1st January and 31st March. As weather-related delays are seasonal, this reduces the validity of conclusion relating to the importance of weather. 

\cite{oneto_et_al_2016} note that weather conditions can additionally influence passenger flow and consequently dwell times, which have already been described as a key influence on delays. 


 \cite{oneto_et_al_2016} CITE ALL HERE use temperature, relative humidity, wind direction, wind speed, rain level, pressure and solar radiation.
\cite{nair_et_al_2019} use weather data from 92 weather observatories, including snow conditions, visibility, and temperature. They found that weather has only a small impact of delays; an analysis of delay-attribution showed that less than 3\% of delays were directly attributed to weather. 

The proposed dataset for this dissertation uses similar fields. For interoperability with forecasts, the comprehensive data provided by the Met Office has been mapped to a more simplistic set of fields: wind gust, relative humidity, visiblity, wind direction, wind speed, temperature, weather type (category), and precipitation probability. 

\cite{wang_et_al_2018} collected weather data from 344 cities along the route in question, Beijing to Guangzhou. It is worth nothing that the two are approximately 2200km, and so delays are of a magnitude not frequently found

\cite{oneto_et_al_2016} found that the inclusion of weather data improved the accuracy of their RF model by approximately 10\%, with the caveat that the further ahead in the future the forecast is (and thus the less accurate, the smaller this increase was. 

\cite{nabian_et_al_2019} only had a daily overview of: maximum wind speed, maximum, minimum, and average temperature, and rain depth. Authors note that the hourly data would have improved the model. Could not be considered significant as a result. 

%INFRASTRUCTURE

\subsection{Infrastructure}

Infrastructure naturally matters to trains. Extant work in ML for TDP has been surprisingly lacking incorporating infrastructure characteristics (single or double track, station layouts, interlocking, cant, speed limit), and so on. It is a thoroughly modelled in the analytical models discussed earlier.

\cite{milinkovic_et_al_2013} groups infrastructure opinions. Collected opinions from traffic dispatchers, operators, and experts famiilar with the functioning of the system. Was used more broadly to define input variables, and the primary causes of delay (not the causes of primary delay). Defined three input parameters: the train category, timetable influence, and the distance travelled by the train. Timetable influence was used as a catch-all of sorts; the study is vague on specifics. It included the influence of infrastructure parameters, timetable characteristics, operation time, the type of locomotive, local conditions, technological solutions, principles for safety and signalling, and weather conditions. This is for the FPN!

For the ANFIS, which used real-life data (go into detail here), an 'infrastructure influence', which included the percentage of restricted speed sections, the number of junctions, and the number of stations). Included section length, section plans, restricted speed, and track routes. 

The authors note that the average track occupancy of a section can indicate possible bottlenecks of a system. This is close to the \textit{tactical} level briefly discussed earlier: the use of data to make decisions on improving infrastructure. 

The dataset for the proposed study includes infrastructure characteristics used by ? to actually plan train delays.

Used the Delphi method. 

It seems inherently obvious is should have a huge effect on the propagation of delays. 

\cite{markovic_et_al_2015} explores an expert opinion in much better defined terms. The influence of multiple factors along a rail line (single-tracking, reduced speeds, characteristics of block and interlocking systems, number of stations, stops, loops, road-rail level crossings, and junctions) is aggregated into one variable with a value determined by the expert opinions of five dispatchers. Estimates obtained via the Delphi method: experts evaluate a route over multiple rounds until a consensus is reached. Strong correlation found between expert opinions and train delays. 

The condition of the Serbian railways is considerably worse than that of many other countries explored. Characterised by: recently renewed lines (enabling maximum speed), lines with sections with TSRs, single and double-track lines, many junctions and railroad crossing, lines split int sections with different signalling and safety equipment. 

Evaluated each route on a scale of 1 - 10. 1 denotes a route with the highest number of infrastructural factors that could cause unplanned delays.

The study only considered a limited number of routes: those passing through Rakovica station. Models for larger areas cannot rely on human assessment of infrastructure conditions. 39 lines were evaluated. 
Specifically: number of stations/stops/junctions/loops/crossings, percentage of single track, percentage with restricted speed, length with restricted speed, block section, track clear section (station distance, braking distance, automatic block system, centralised traffic control, axle counters).

\cite{nair_et_al_2019} take exactly this approach. The authors reconstruct the network and estimate capacity directly from passing messages. Furthermore, the method used generated train-class specific networks. The inferred method is employed for various downstream tasks: inferring train paths, conflict status estimation, typical travel time estimation. "Passing" messages are sorted by date, time, and train. If there are sufficient observations, the control point and track stretch is recorded as an edge. The frequency of transitions from each outgoing edge, the mean and standard deviation of travel times are also recorded for each edge. A feasibility matrix for each outgoing edge is recorded at each vertex, which records pairwise edge feasible flows at each section by identifying movements by two trains in a short time window; this is used to identify potential conflicts between trains when there are deviations from the schedule. Reconstructed networks around several major hubs were inspected by hand and found to be accurate. 

Station attributes used included the designated platform, station attributes, historical mean delay at tracks, platforms, actual platform, track allocation, and track / platform change status.

% Maintenance

\subsection{Maintenance}

Only one paper was found that incorporated the "maintenance" class: \cite{nair_et_al_2019}. The authors used work zone information, indicating location, duration, and the likely impact of different train categories.
The proposed dataset uses the RDG Knowledgebase API, which includes data on "incidents": service disruptions and engineering works.

% Other

\subsection{Other}

No papers were found to incorporate accidents, vandalism, trespassing, or fatalities, or strikes. Holidays are, however, included in \cite{nair_et_al_2019}, and will be included in the proposed dataset. NR categorises the day by weekday, Saturday, Sunday, Christmas, and Bank holiday, reflecting the different timetables used for each.







\clearpage
\onecolumn

\printbibliography
 
\end{document}