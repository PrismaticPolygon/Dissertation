% !TEX TS-program = pdflatex
% !TEX encoding = UTF-8 Unicode

% This is a simple template for a LaTeX document using the "article" class.
% See "book", "report", "letter" for other types of document.

\documentclass[11pt]{article} % use larger type; default would be 10pt

\usepackage[utf8]{inputenc} % set input encoding (not needed with XeLaTeX)

%%% Examples of Article customizations
% These packages are optional, depending whether you want the features they provide.
% See the LaTeX Companion or other references for full information.

%%% PAGE DIMENSIONS
\usepackage{geometry} % to change the page dimensions
\geometry{a4paper} % or letterpaper (US) or a5paper or....
% \geometry{margin=2in} % for example, change the margins to 2 inches all round
% \geometry{landscape} % set up the page for landscape
%   read geometry.pdf for detailed page layout information

\usepackage{graphicx} % support the \includegraphics command and options

% \usepackage[parfill]{parskip} % Activate to begin paragraphs with an empty line rather than an indent

%%% PACKAGES
\usepackage{booktabs} % for much better looking tables
\usepackage{array} % for better arrays (eg matrices) in maths
\usepackage{paralist} % very flexible & customisable lists (eg. enumerate/itemize, etc.)
\usepackage{verbatim} % adds environment for commenting out blocks of text & for better verbatim
\usepackage{subfig} % make it possible to include more than one captioned figure/table in a single float
% These packages are all incorporated in the memoir class to one degree or another...

%%% HEADERS & FOOTERS
\usepackage{fancyhdr} % This should be set AFTER setting up the page geometry
\pagestyle{fancy} % options: empty , plain , fancy
\renewcommand{\headrulewidth}{0pt} % customise the layout...
\lhead{}\chead{}\rhead{}
\lfoot{}\cfoot{\thepage}\rfoot{}

%%% SECTION TITLE APPEARANCE
\usepackage{sectsty}
\allsectionsfont{\sffamily\mdseries\upshape} % (See the fntguide.pdf for font help)
% (This matches ConTeXt defaults)

%%% ToC (table of contents) APPEARANCE
\usepackage[nottoc,notlof,notlot]{tocbibind} % Put the bibliography in the ToC
\usepackage[titles,subfigure]{tocloft} % Alter the style of the Table of Contents
\renewcommand{\cftsecfont}{\rmfamily\mdseries\upshape}
\renewcommand{\cftsecpagefont}{\rmfamily\mdseries\upshape} % No bold!

%%% END Article customizations

%%% The "real" document content comes below...

\title{Dissertation project plan}
\author{Dominic White}
%\date{} % Activate to display a given date or no date (if empty),
         % otherwise the current date is printed 

\begin{document}
\maketitle

\section*{Description}

Although considerable work has been done for real-time train delay prediction, the timeframe considered by this project (medium-term; up to 5 days’ in advance) is poorly, if at all, explored. This project will therefore be based on real-time work and suitably adapted. A machine learning system will be developed using weather data, historic train delay data, and various other data for this purpose. An API will be developed to expose the model to forecast data and unseen schedules, with the ultimate objective of a simple application.

\section*{Preliminary preparation}

\begin{itemize}

	\item{Sourcing suitable datasets}
	\item{Merging said datasets}

\end{itemize}

\section*{Objectives}

\subsection*{Minimum}

\begin{itemize}

	\item{Implementation of a machine learning system}

\end{itemize}

\subsection*{Intermediate}

\begin{itemize}

	\item{Hosted API to expose model to unseen scheules and weather forecast data}

\end{itemize}

\subsection*{Advanced}

\begin{itemize}

	\item{User interface to access predictions via API}

\end{itemize}

\section*{Project plan}

Although the preliminary preparations are of great importance (as for any machine learning, the data is vital), they cannot in of themselves constitute a deliverable. Considerable time will therefore be dedicated to these preparations. To ensure maximum scope for exploration, as little pre-processing as possible will be performed prior to the start of the minimum objective. 
Most of the time spent on this project will be on the minimum objective. \\

The completion of the intermediate objective should be fairly easy – the dataset will be formatted to match both schedule data and forecast data. However, the amalgamation of these various APIs, as well as hosting an API exposing the trained model, will likely be time-consuming. It is likely in the course of this objective the work of the previous objective will have to be modified slightly. \\

The completion of the advanced objective is likely unrestrained. For the system to be useful, consumers must be able to access delay predictions through a convenient interface. This will likely take the form of either a hybrid application or website, which allows the user to look up train routes (itself a complex problem) and then predicts the likelihood of delays for those routes. For actual value, the system would also have to incorporate a booking system, or delegate this to an API, but this is beyond the scope of the project at this stage. 


\end{document}
